
\subsubsection*{SQL-Befehle}
\begin{verse}
\begin{tabular}{>{\raggedright}p{3cm}|>{\raggedright}p{6cm}|>{\raggedright}p{7cm}|}
\multicolumn{1}{>{\raggedright}p{3cm}}{} & \multicolumn{1}{>{\raggedright}p{6cm}}{\textbf{SQL}} & \multicolumn{1}{>{\raggedright}p{7cm}}{\textbf{Algebraische Schreibweise / Beschreibung}}\tabularnewline
\cline{2-3} 
\textbf{SELECT, FROM WHERE} & SELECT Name, Vorname FROM Besucher & $\Pi_{Nam,Vorname}(Besucher)$\tabularnewline
\cline{2-3} 
 & SELECT Name, Vorname FROM Besucher WHERE Name='Müller' & $\Pi_{Nam,Vorname}(\sigma_{Name='M\ddot{u}ller'}(Besucher))$\tabularnewline
\cline{2-3} 
\textbf{LIKE} & WHERE Name LIKE 'Pe\%' & Alle Name die mit 'Pe' beginnen\tabularnewline
\cline{2-3} 
 & WHERE name LIKE '\_da\%' & Alle Namen die 'da' nach dem ersten Zeichen enthalten\tabularnewline
\cline{2-3} 
\textbf{BETWEEN} & WHERE Name BETWEEN 'Bucher' AND 'Schmid' & Alle Name zwischen Bucher und Schmid\tabularnewline
\cline{2-3} 
 & WHERE Hausnummer BETWEEN 1 AND 10 & Alle Hausnummern zwischen 1 und 10 (inkl. 1 und 10!)\tabularnewline
\cline{2-3} 
\textbf{AS (Scope variables)} & SELECT x.Name, x.Vorname, y.Ort FROM Besucher AS x, Gast AS y WHERE
x.PLZ = y.PLZ & AS nicht unbedingt nötig, alternativ geht auch FROM Besucher x, Gast
y\tabularnewline
\cline{2-3} 
\textbf{(NOT) EXISTS} & SELECT x.Name, x.Vorname, x.Ort FROM Besucher AS x WHERE NOT EXISTS
(SELECT y.Vorname, y.Name FROM Besucher AS y WHERE y.Name = y.Name
AND x.Vorname < y.Vorname) & ALLE Besucher, ausser die, deren Nachnamen nochmals vorkommt, von
denen wird der, mit dem grössten (``höchste'') Vornamen genommen. \tabularnewline
\cline{2-3} 
\textbf{SUM} & SELECT SUM(Lagerbestand) FROM Sortiment & Summe aller Werte von ``Lagerbestand''\tabularnewline
\cline{2-3} 
 & SELECT SUM (DISTINCT Lagerbestand) FROM Sortiment & Summe aller Werte ohne Duplikate\tabularnewline
\cline{2-3} 
\textbf{COUNT} & SELECT COUNT({*}) FROM Besucher & Zählt alle Tupel in der Tabelle Besucher\tabularnewline
\cline{2-3} 
 & SELECT COUNT(DISTINCT {*}) FROM Artikel & Alle Tupel, Ohne Duplikate\tabularnewline
\cline{2-3} 
 & SELECT COUNT(Name,Vorname) & Geht nicht, stattdessen:\tabularnewline
\cline{2-3} 
 & SELECT COUNT({*}) FROM (SELECT Name, Vorname From Besucher) & \tabularnewline
\cline{2-3} 
\textbf{GROUP BY} & SELECT Restaurant, MAX(Frequenz) FROM Gast GROUP BY Restaurant & Höchste Frequenz eines Restaurants\tabularnewline
\cline{2-3} 
 & SELECT Restaurant

FROM Gast GROUP BY Restaurant

HAVING COUNT({*}) > 5 & Alle Restaurants die mehr als 5 Gäste haben\tabularnewline
\cline{2-3} 
 & SELECT x.Restaurant FROM Gast AS x WHERE 5 < ( SELECT COUNT({*}) FROM
Gast AS y WHERE y.Restaurant = x.Restaurant) & Gleiches Statement, ohne GROUP BY\tabularnewline
\cline{2-3} 
\end{tabular}
\end{verse}

\subsubsection*{CREATE}

CREATE TABLE Name 

(

A1 INTEGER NOT NULL,

A2 INTEGER NOT NULL,

Attr1Name VARCHAR(30) NOT NULL,

Attr2Name DATE NOT NULL, 

Attr3Name TIME NOT NULL, 

Attr4Name TIMESTAMP NOT NULL,

PRIMARY KEY (A1),

FOREIGN KEY (A2) REFERENCES TABLE TableName,

UNIQUE (A1,...)

)
