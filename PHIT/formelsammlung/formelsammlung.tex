\documentclass[11pt]{article}

\usepackage[utf8x]{inputenc}

\title{Formelsammlung}
\date{\today}
\author{Christian Brüesch}

\begin{document}
\maketitle
\section{Kapitel 1, Geschwindigkeit, Strecke, Beschleunigung}
$$v(t) = at + v_0$$
$$s(t) = \frac{1}{2}at^2 + v_0t + s_0$$
\section{Kapitel 1, Der vertikale und der schiefe Wurf}
\subsection{Erdanziehung}
$$m\overrightarrow{a} = \overrightarrow{F}$$\\
$$\overrightarrow{F}=\left(\begin{array}{c}
0\\
0\\
-mg
\end{array}\right)$$
\subsection{Der Wurf}
\subsubsection{Beschleunigung}
$$\overrightarrow{a} = \left( \begin{array}{c}
0\\
0\\
-g
\end{array} \right)$$
\subsubsection{Geschwindigkeit}
$$\overrightarrow{v}(t) = \left( \begin{array}{c}
v_{x,0}\\
v_{y,0}\\
v_{z,0} - gt
\end{array} \right) $$
\subsubsection{Ortsvektor}
$$\overrightarrow{r}(t) = \left( \begin{array}{c}
r_{x,0} + v_{x,0}t\\
r_{y,0} + v_{y,0}t\\
r_{z,0} + v_{z,0}t - g\frac{t^2}{2}
\end{array} \right) $$
Üblicherweise setzt man $r_{x,0} = r_{y,0} = 0$.
\subsection{Kreisbewegnung}
$$v = r\omega$$
$$a = r\omega^2$$
\end{document}