\documentclass[11pt]{article}

\usepackage[utf8x]{inputenc}
\usepackage{fullpage}

\title{Formelsammlung}
\date{\today}
\author{Christian Brüesch}

\begin{document}
\tableofcontents
\maketitle
\section{Geschwindigkeit, Strecke, Beschleunigung}
$$v(t) = at + v_0$$
$$s(t) = \frac{1}{2}at^2 + v_0t + s_0$$
\section{Der vertikale und der schiefe Wurf}
\subsection{Erdanziehung}
$$m\overrightarrow{a} = \overrightarrow{F}$$\\
$$\overrightarrow{F}=\left(\begin{array}{c}
0\\
0\\
-mg
\end{array}\right)$$
\subsection{Beschleunigung}
$$\overrightarrow{a} = \left( \begin{array}{c}
0\\
0\\
-g
\end{array} \right)$$
\subsection{Geschwindigkeit}
$$\overrightarrow{v}(t) = \left( \begin{array}{c}
v_{x,0}\\
v_{y,0}\\
v_{z,0} - gt
\end{array} \right) $$
\subsection{Ortsvektor}
$$\overrightarrow{r}(t) = \left( \begin{array}{c}
r_{x,0} + v_{x,0}t\\
r_{y,0} + v_{y,0}t\\
r_{z,0} + v_{z,0}t - g\frac{t^2}{2}
\end{array} \right) $$
Üblicherweise setzt man $r_{x,0} = r_{y,0} = 0$.
\subsection{Kreisbewegnung}
$$v = r\omega$$
$$a = r\omega^2$$
\section{Kräfte}
\subsection{Anziehung zwischen zwei Objekten}
$$\overrightarrow{F}_{mM} = -\gamma * \frac{mM}{|\overrightarrow{r}_m-\overrightarrow{r}_M|^2} * \frac{\overrightarrow{r}_m-\overrightarrow{r}_M}{|\overrightarrow{r}_m-\overrightarrow{r}_M|^2} $$

\begin{eqnarray*}
\overrightarrow{F}_{mM} &:& Anziehungskraft\\
\gamma &:& Konstante:6.67*10^{-11} m^3kg^{-1}s^{-2}\\
-\gamma &:& \textit{Minus, da es sich um eine Anziehung handelt.}\\
\frac{\overrightarrow{r}_m-\overrightarrow{r}_M}{|\overrightarrow{r}_m-\overrightarrow{r}_M|^2}&:&Richtung\\
|\overrightarrow{r}_m-\overrightarrow{r}_M| &:& \sqrt{(r_x)^2 + (r_y)^2 + (r_z)^2}
\end{eqnarray*}
\subsection{Umlaufbahnberechnungen}
\begin{eqnarray*}
r &=& r_E + h\\
m\omega^2r &=& \gamma\frac{mM}{r^2} \textit{(m kürzt sich raus)}\\
\omega &=& 2\pi\nu = \frac{2\pi}{T}\\
T &=& 24h * 3600 = 86400s \textit{(Frequenz einer Umdrehung der Erde)}
\end{eqnarray*}
\subsection{Anziehung zwischen Atomen}
\begin{eqnarray*}
\overrightarrow{F}_C&=&\frac{1}{4\pi\epsilon_0}*\frac{qQ}{r^2} \\
\\
E_0 &:& Konstante: 8.859*10^{-12} \frac{C^2}{Jm}\\
q &:& \textit{Gewicht. Wie m bei Objekten}\\
Q &:& \textit{Gewicht. Wie M bei Objekten}\\
r &:& \textit{Abstand zwischen den Atomen}\\
\end{eqnarray*}
\subsection{Herleitung Zentripetalkraft}
\begin{eqnarray*}
v&=&r\omega\\
a&=&r\omega^2\\
\overrightarrow{F} &=& mg = ma\\
ma &=& mr\omega^2\\
ma &=& mr\frac{v^2}{r^2} = \frac{mv^2}{r}\\
\end{eqnarray*}
\subsection{Reibungskräfte}
\subsubsection{Objekt auf schiefer Ebene}
\begin{eqnarray*}
F_g &=& F_N + F_R\\
F_R &=& F_g * \sin(\alpha) = mg*\sin(\alpha)\\
F_N &=& F_g * \cos(\alpha) = mg*\cos(\alpha)\\
\textit{Haftung}&:& 0 \le F_H \le \mu_H * F_N\\\\
F_{R_{max}} &=& \mu_H * FN\\
mg * sin(\alpha_{max}) &=& \mu_H * mg * cos(\alpha_{max})
\end{eqnarray*}
\subsubsection{Motorrad fährt in Kurve}
\begin{eqnarray*}
\textit{Ansatz}&:& \textit{Haftreibung} > \textit{Zentripetalkraft}\\
\mu_H * mg &>& \frac{mv^2}{r}\\
\end{eqnarray*}

\end{document}