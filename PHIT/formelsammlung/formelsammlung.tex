\documentclass[11pt]{article}

\usepackage[utf8x]{inputenc}
\usepackage{fullpage}
\usepackage{amsmath}
\usepackage{graphicx}

\title{Formelsammlung}
\date{\today}
\author{Christian Brüesch}

\begin{document}
\tableofcontents
\maketitle
\section{Geschwindigkeit, Strecke, Beschleunigung}
$$v(t) = at + v_0$$
$$s(t) = \frac{1}{2}at^2 + v_0t + s_0$$
\section{Der vertikale und der schiefe Wurf}
\subsection{Erdanziehung}
$$m\overrightarrow{a} = \overrightarrow{F}$$\\
$$\overrightarrow{F}=\left(\begin{array}{c}
0\\
0\\
-mg
\end{array}\right)$$
\subsection{Beschleunigung}
$$\overrightarrow{a} = \left( \begin{array}{c}
0\\
0\\
-g
\end{array} \right)$$
\subsection{Geschwindigkeit}
$$\overrightarrow{v}(t) = \left( \begin{array}{c}
v_{x,0}\\
v_{y,0}\\
v_{z,0} - gt
\end{array} \right) $$
\subsection{Ortsvektor}
$$\overrightarrow{r}(t) = \left( \begin{array}{c}
r_{x,0} + v_{x,0}t\\
r_{y,0} + v_{y,0}t\\
r_{z,0} + v_{z,0}t - g\frac{t^2}{2}
\end{array} \right) $$
Üblicherweise setzt man $r_{x,0} = r_{y,0} = 0$.
\subsection{Kreisbewegnung}
$$v = r\omega$$
$$a = r\omega^2$$
\section{Kräfte}
\subsection{Anziehung zwischen zwei Objekten}
$$\overrightarrow{F}_{mM} = -\gamma * \frac{mM}{|\overrightarrow{r}_m-\overrightarrow{r}_M|^2} * \frac{\overrightarrow{r}_m-\overrightarrow{r}_M}{|\overrightarrow{r}_m-\overrightarrow{r}_M|} $$

\begin{eqnarray*}
\overrightarrow{F}_{mM} &:& Anziehungskraft\\
\gamma &:& Konstante:6.67*10^{-11} m^3kg^{-1}s^{-2}\\
-\gamma &:& \textit{Minus, da es sich um eine Anziehung handelt.}\\
\frac{\overrightarrow{r}_m-\overrightarrow{r}_M}{|\overrightarrow{r}_m-\overrightarrow{r}_M|}&:&Richtung\\
|\overrightarrow{r}_m-\overrightarrow{r}_M| &:& \sqrt{(r_x)^2 + (r_y)^2 + (r_z)^2}
\end{eqnarray*}
\subsection{Umlaufbahnberechnungen}
\begin{eqnarray*}
r &=& r_E + h\\
m\omega^2r &=& \gamma\frac{mM}{r^2} \textit{(m kürzt sich raus)}\\
\omega &=& 2\pi\nu = \frac{2\pi}{T}\\
T &=& 24h * 3600 = 86400s \textit{(Frequenz einer Umdrehung der Erde)}
\end{eqnarray*}
\subsection{Anziehung zwischen Atomen}
\begin{eqnarray*}
\overrightarrow{F}_C&=&\frac{1}{4\pi\epsilon_0}*\frac{qQ}{r^2} \\
\\
E_0 &:& Influenzkonstante: 8.859*10^{-12} \frac{C^2}{Jm}\\
q &:& \textit{Gewicht. Wie m bei Objekten}\\
Q &:& \textit{Gewicht. Wie M bei Objekten}\\
r &:& \textit{Abstand zwischen den Atomen}\\
\end{eqnarray*}
\subsection{Herleitung Zentripetalkraft}
\begin{eqnarray*}
v&=&r\omega\\
a&=&r\omega^2\\
\overrightarrow{F} &=& mg = ma\\
ma &=& mr\omega^2\\
ma &=& mr\frac{v^2}{r^2} = \frac{mv^2}{r}\\
\end{eqnarray*}
\subsection{Reibungskräfte}
\subsubsection{Objekt auf schiefer Ebene}
\begin{eqnarray*}
F_g &=& F_N + F_R\\
F_R &=& F_g * \sin(\alpha) = mg*\sin(\alpha)\\
F_N &=& F_g * \cos(\alpha) = mg*\cos(\alpha)\\
\textit{Haftung}&:& 0 \le F_H \le \mu_H * F_N\\\\
F_{R_{max}} &=& \mu_H * F_N\\
mg * sin(\alpha_{max}) &=& \mu_H * mg * cos(\alpha_{max})
\end{eqnarray*}
\subsubsection{Motorrad fährt in Kurve}
\begin{eqnarray*}
\textit{Ansatz}&:& \textit{Haftreibung} > \textit{Zentripetalkraft}\\
\mu_H * mg &>& \frac{mv^2}{r}\\
\end{eqnarray*}
\section{Impuls und Kraft}
\subsubsection{Zentripetalbeschleunigung}
\begin{eqnarray*}
F_z &=& m\cdot\frac{v^2}{r} = m \cdot a_z\\
a_z &=& \frac{v^2}{r}\\
F_z &=& mr\omega^2\\
\omega &=& \textit{Winkelgeschwindigkeit}
\end{eqnarray*}


\subsubsection{Impuls}
\begin{eqnarray*}
\overrightarrow{p}&=&m\overrightarrow{v}\\
\textit{Einheit }&:& [kg\cdot m/s]\\
F &=& \frac{d\overrightarrow{p}}{dt} \\
\overrightarrow{F} &=& m\frac{d\overrightarrow{v}}{dt}+\overrightarrow{v}\frac{dm}{dt}\\
\overrightarrow{F} &=& \frac{d\overrightarrow{p}}{dt}= m\frac{d\overrightarrow{v}}{dt}=m\overrightarrow{a}\\
\overrightarrow{p} &=& \overrightarrow{p}_A+\overrightarrow{p}_B = const.\\
\overrightarrow{v}&=&\frac{m_A\overrightarrow{v}_A+m_B\overrightarrow{v}_B}{m_A+m_B}\\
m_1\frac{u_1^2}{2}+m_2\frac{u_2^2}{2} &=& m_1\frac{v_1^2}{2}+ m_2\frac{v_2^2}{2} = E\\
E_{Kin} &=& \frac{m\overrightarrow{v}^2}{2} = \frac{\overrightarrow{p}\overrightarrow{v}}{2} = \frac{\overrightarrow{p}^2}{2m}
\end{eqnarray*}


\subsubsection{Sprung von Wagen}
Sie stehen auf einem Wagen. Sie und der Wagen wiegen zusammen M = 300kg und ste-
hen still. Sie springen vom Wagen in Richtung gegeben durch einen normierten Rich-
tungsvektor n. Bei Ihrem Sprung erreichen Sie eine Schnelligkeit von 20m/s. Mit welcher
Geschwindigkeit (als Vektor!) bewegt sich der Wagen?
\begin{eqnarray*}
\textit{Vor dem Sprung ist der Impuls 0: } \overrightarrow{p}_{in}&=& 0\\
\textit{Ihr Impuls wird sein: } \overrightarrow{p}_{Sie}&=&m_{Sie}vm\\
\textit{Nun ist der Impuls erhalten: } \overrightarrow{p}_{Sie} + \overrightarrow{p}_{Wagen} = \overrightarrow{p}_{in} &=& 0\\
\textit{Mit } \overrightarrow{p}_{Wagen} &=& m_{Wagen} v_{Wagen} \textit{ergibt sich:}\\
\overrightarrow{v}&=& -\frac{m_{Sie}}{M-m_{Sie}}v\overrightarrow{n}
\end{eqnarray*}
\subsubsection{Aufgabe mit Wind auf schräge Platte}

\begin{eqnarray*}
\overrightarrow{F} = \begin{pmatrix}
						(1-\sin(\alpha))\sin(\alpha)\rho_{air}Av^2\\
						0\\
						-\cos(\alpha)\sin(\alpha)\rho_{air}Av^2
					\end{pmatrix}	
\end{eqnarray*}

\section{Gravitationskraft}
\begin{eqnarray*}
F_G &=& \gamma \frac{mM}{r^2}\\
\gamma &=& 6.6731\cdot 10^{-11} Nm^2/kg^2\\
\overrightarrow{F}_{Mm} &=& -\overrightarrow{F}_{mM}
\end{eqnarray*}

\section{Elektrische Kräfte}
\begin{eqnarray*}
F_E &=& \frac{1}{4\pi\epsilon_0}\cdot\frac{qQ}{r^2}\\
\epsilon_0 &=& 8.8542\cdot 10^{-12}C^2/(Nm^2) \textit{(Elektrische Feldkonstante)}
\end{eqnarray*}

\section{Fall- und Wurfbewegungen mit Luftwiderstand}
\begin{eqnarray*}
F_w&=& c_w \cdot \frac{\rho A}{2}*v^2\\
\rho &=& 1.293 kg/m^3 \textit{(Standardatmosphäre)}\\
A &=& \textit{Querschnittsfläche senkrecht zur Anströmung stehend}\\
v &=& v(t) = [(c_w \cdot \frac{\rho A}{2m})\cdot t - c]^{-1}\\
v_{n+1} &=& v_n+\delta v_n = v_n + (g-c_w\cdot\frac{\rho A}{2m} \cdot v_n^2)*\delta t
\end{eqnarray*}

\section{Energie}
Einheit von Energie: Newtonmeter [Nm] oder Joule [J] (1J = 1Nm)
\subsection{Potentielle Energie}
\begin{eqnarray*}
E_{pot} &=& W = F\cdot s = mgh \textit{  Gilt nur auf der Erde}\\
E_{pot} &=& -\frac{\gamma Mm}{r}\\
\gamma &:& 6.67 \cdot 10^{-11} Nm^2/kg^2\\
M &:& \textit{Masse der Erde (oder Sonne) }\textit{Masse Erde: }5,794\cdot 10^{24} kg\\
m &:& \textit{Masse des Objektes}\\
r &:& \textit{Abstand zwischen Erdmittelpunkt zu Objekt  Radius Erde: } 6,371\cdot 10^6 m\\
\end{eqnarray*}
\textit{Falls der Radius undendlich ist, wird der Radius der Erde verwendet.}
\subsection{Kinetische Energie}
\begin{eqnarray*}
E_{Kin} &=& \frac{1}{2}mv^2\\
m &:& \textit{Masse}\\
v &:& \textit{Geschwindigkeit}
\end{eqnarray*}

\subsection{Federenergie}
\begin{eqnarray*}
E_{Feder} &=& \frac{k(x-x_0)^2}{2}\\
k &:& \textit{Federkonstante}\\
x_0 &:& \textit{Anfangsposition}\\
x &:& \textit{Endposition}
\end{eqnarray*}

\subsection{Energie Kugel}
$$\frac{1}{4\pi \epsilon_0}\frac{Q}{r_{Q_q}^2}$$
\section{Leistung}
Einheit von Leistung P: Watt [W] oder Joule pro Sekunde [J/s] (1W = 1J/s)
\begin{eqnarray*}
P &=& F \cdot v\\
P &=& \frac{W}{t}
\end{eqnarray*}

\subsection{Elektrische Energie}
$$U_{AB}=-\frac{Q}{4\pi \epsilon_0} \int_A^B \frac{1}{r^2} = \frac{Q}{4\pi \epsilon_0}\cdot [\frac{1}{r_B}-\frac{1}{r_A}]$$
$$E_{elek} = E_{Kin}$$

\section{Elektrisches Feld}
\begin{eqnarray*}
\overrightarrow{E}&=&\frac{\overrightarrow{F}_E}{Q}\\
Q &:& \textit{Gespeicherte Ladung im Kondensator}\\
\overrightarrow{F}_E &:& \textit{Elektrische Kraft (siehe 6)}\\
\overrightarrow{E} &=& \textit{Elektrisches Feld}
\end{eqnarray*}

\section{Gradient}
\begin{eqnarray*}
\textit{Geg }&:&xy^2z^3\\
\textit{Gradient }&:& (\frac{y^2z^3}{\frac{2xyz^3}{3xy^2z^2}})\textit{(Ableiten auf allen drei Ebenen)}
\end{eqnarray*}

\section{Dipol}
\begin{eqnarray*}
\overrightarrow{p} &=& q\overrightarrow{l}\\
\textit{Potential} &:& q(\overrightarrow{r}) = \frac{1}{4\pi \epsilon_0}\cdots\\
\end{eqnarray*}
Kugel: Punktladung\\
Zylinder: Draht\\
Platte: Kondensator



\section{Kapazität}
\begin{eqnarray*}
CU &=& Q\\
E_{elek} &=& \frac{CU^2}{2}\\
U &:& \textit{Spannung}\\
C &:& \textit{Kapazität}\\
Q &:& \textit{Ladung}
\end{eqnarray*}


\section{Ströme}
\begin{eqnarray*}
P &=& U\cdot I\\
E_{pot} &=& \frac{Q^2}{2C} = C \frac{U^2}{2}\\
U &:& \textit{Spannung}\\
I &:& \textit{Strom}\\
C &:& \textit{Kapazität}\\
Q &:& \textit{Ladung}
\end{eqnarray*}

\section{Kondensatoren}
\subsection{Serieschaltung}
$$\frac{1}{C_g}=\frac{1}{C_1}+\frac{1}{C_2}$$


\subsection{Parallelschaltung}
$$ C_g = C_1 +C_2$$

\section{Widerstände}
\subsection{Serieschaltung}
$$R_g=R_1+R_2$$
\begin{eqnarray*}
IR_1&=&U_1 \textit{(Ohm'sches Gesetz)}\\
IR_2&=&U_2 \textit{(Ohm'sches Gesetz)}\\
U_0 - U_1 - U_2 &=& 0 \textit{(Maschenregel)}\\
IR &=& U \textit{(Ohm'sches Gesetz)}\\
U_0 - U &=& 0 \textit{(Maschenregel)}\\\\
I(R_1+R_2)&=&U_1+U_2 = U = IR -> (R_1+R_2) = R
\end{eqnarray*}
\includegraphics[scale=0.7]{widerstandSerie.jpg}
\pagebreak
\subsection{Parallelschaltung}
$$R_g= \frac{1}{\frac{1}{R_1}+\frac{1}{R_2}}$$
\begin{eqnarray*}
I_1R_1 &=& U_1 \textit{(Ohm'sches Gesetz)}\\
I_3R_2 &=& U_2 \textit{(Ohm'sches Gesetz)}\\
U_1 &=& U_2 \textit{(Maschensatz)}\\
I_0 &=& I_1 + I_2 \textit{(Knotensatz)}\\
IR &=& U \textit{(Ohm'sches Gesetz)}\\
U_0 - U &=& 0 \textit{(Maschenregel)}\\\\
I_1+I_2 &=& \frac{U_1}{R_1}+\frac{U_2}{R_2} -> \frac{U}{R}=I = I_0 = U_1(\frac{1}{R_1}+\frac{1}{R_2} -> \frac{1}{R} = (\frac{1}{R_1}+\frac{1}{R_2})
\end{eqnarray*}
\includegraphics[scale=0.7]{widerstandParallel.jpg}

\section{Lorenzkraft}
\begin{eqnarray*}
\overrightarrow{F}_B&=&q(\overrightarrow{v}X\overrightarrow{B})\\\\
\overrightarrow{F}_B &:& \textit{Lorenzkraft}\\
\overrightarrow{v}&:& \textit{Geschwindigkeit einer positiven Ladung}\\
\overrightarrow{B}&:& \textit{Magnetfeld}\\
\textit{Einheit}&:& \textit{Tesla } (N\cdot s /(C\cdot m) = T)
\end{eqnarray*} 

\begin{eqnarray*}
\overrightarrow{F}_B&=&q(\overrightarrow{E}+\overrightarrow{v}X\overrightarrow{B})\\\\
\overrightarrow{E}&:& \textit{Elektrische Feld}\\
\overrightarrow{F}_B &:& \textit{Lorenzkraft}\\
\overrightarrow{v}&:& \textit{Geschwindigkeit einer positiven Ladung}\\
\overrightarrow{B}&:& \textit{Magnetfeld}\\
\textit{Einheit}&:& \textit{Tesla } (N\cdot s /(C\cdot m) = T)
\end{eqnarray*}
\end{document}
