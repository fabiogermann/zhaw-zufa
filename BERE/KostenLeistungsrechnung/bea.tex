
\subsection*{Break-Even-Analyse}

Basiert auf der Spaltung von fixen und variabeln Kosten. Man geht
bei der BE-Analyse davon aus, dass Produktions- und Verkaufsmenge
identisch sind.


\subsubsection*{Mengenmässige BEA}
\begin{itemize}
\item Deckungsbeitrag = Preis - variable Kosten (pro Mengeneinheit)
\item Gewinn = Erlös - Kosten
\item Breakeven = $\frac{Fixe\, Kosten}{Deckungsbeitrag}$ = Anzahl Produkte
\item Breakeven mit Gewinn X = $\frac{Fixe\, Kosten+erw\ddot{u}nschter\, Gewinn}{Deckungsbeitrag}$=
Anzahl Produkte
\end{itemize}
\begin{tabular}{|c|c|c|c|}
\hline 
$C=Totalkosten$ & $F=fixe\, Kosten$ & $v=var.\; Kosten\, je\, Mengeneinheit\,(ME)$ & $Q=Produktionsvolumen\, in\, ME$\tabularnewline
\hline 
\hline 
$C=F+(v\times Q)$  & $Erl\ddot{o}s=E=(p\times Q)$  & $Gewinn=E-K=((p-v)\times Q)-F$ & $Deckungsbeitrag=(p-v)=Preis-var.\, Kosten$\tabularnewline
\hline 
\end{tabular}

Um den Breakeven Punkt $Q^{0}$ zu bestimmen, setzten wir den Gewinn
gleich NULL. $0=((p-v)\times Q^{0})-F\Longrightarrow Q^{0}=\frac{F}{p-v}$


\subsubsection*{Zielgewinnbestimmung}

BEA kann einfach erweitert werden um festzustellen bei welcher Produktionsmenge
der gewüneschte Gewinn erreicht wird.

$Zielgewinn\,(T)=[(p-v)\times Q^{T}]-F\Longrightarrow(p-v)\times Q^{T}=F+T\Longrightarrow Q^{T}=\frac{F+T}{p-v}$


\subsubsection*{Wertmässige Breakeven Analyse}

Jemand plant Glaces zu verkaufen. Plan: Preis ist 25\% über var. Kosten.
Standausrüstung plus Miete: 800.- => wie gross muss der Umsatz sein?
(Preis wurde angenommen)

\[
Erl\ddot{o}s=1.25\times var.\, Kosten\Longrightarrow Var.\, Kosten=0.8\times Erl\ddot{o}s
\]



\subsubsection*{
\[
Q^{0}=\frac{Fixe\, Kosten}{Deckungsbetrag\, je\, ME}=\frac{800}{(1.00-0.80)\, je\, ME}=800/0.20=4000\, ME
\]
 Kurzfristige Preisuntergrenze}

Solange sich mit dem Verkauf einer Leistun. ein positiver Deckungsbeitrag
erziehlen lässt, steuert sein Absatz einen Beitrag zur deckung der
fixen Kosten, und vergrössert deshalb den Gewinn. (BSP: Nicht voll
ausgeschöpfte Kapazität mittels preisgünstigen Angeboten: StandByTickets)


\subsubsection*{Sortimentspolitik}

Man soll nur Leistungen im Sortiment führen, welche einen positiven
Deckungsbeitrag abwerfen. Verfügt ein Betrieb über nicht ausgelastete
Kapazität, soll die Leistung(en) gefördert werden, welche den höchsten
Deckungsbeitrag je ME liefern. Liegt ein Kapazitätsengpass vor, so
soll die Leistung mit dem grössten Deckungsbeitrag präferiert werden.
