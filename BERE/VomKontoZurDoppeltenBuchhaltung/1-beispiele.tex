
\part*{Beispiele - Konto \& Buchhaltung}


\section*{Buchungssätze (Sammlung)}
\begin{verse}
\begin{tabular}{|l|r|c|l|r|}
\hline 
Text & Soll &  & Haben & Betrag\tabularnewline
\hline 
\hline 
Barbezug vom Postkonto & Kasse & / & Postkonto & 100\tabularnewline
\hline 
\hline 
Buchert der Fahrzeuge & Fahrzeuge & / & Bilanz & 37'400\tabularnewline
\hline 
Kauf neuer Lieferwagen & Fahrzeuge & / & Kreditoren & 72'000\tabularnewline
\hline 
Eintausch alter Lieferwagen & Kreditor & / & Fahrzeuge & 4'500\tabularnewline
\hline 
(Rechnungsbetrag) & - & / & - & 67'500\tabularnewline
\hline 
(Buchwert alter Lieferwagen) & - & / & - & 10'500\tabularnewline
\hline 
Buchverlust & Abschreibungen & / & Fahrzeuge & 6'000\tabularnewline
\hline 
Skontoabzug 1\% & Kreditor & / & Fahrzeuge & 675\tabularnewline
\hline 
Zahlung der Rechnung & Kreditor & / & Bank & 66'825\tabularnewline
\hline 
Abschreibungen der Fhrz & Abschreibungen & / & Fahrzeuge & 24'500\tabularnewline
\hline 
\hline 
Eröffnungsbuchung & Fahrzeuge & / & Bilanz & 37'400\tabularnewline
\hline 
Abschlussbuchung & Bilanz & / & Fahrzeuge & 73'725\tabularnewline
\hline 
\end{tabular}
\end{verse}
Konto Fahrzeuge:
\begin{verse}
\begin{tabular}{lr|lr}
Soll & \multicolumn{2}{c}{Fahrzeuge} & Haben\tabularnewline
\hline 
Buchwert AB & 37'400 & Eintausch LW & 4'500\tabularnewline
neuer LW & 72'000 & Buchverlust & 6'000\tabularnewline
 &  & Skontoabzug & 675\tabularnewline
 &  & Abschreibungen & 24'500\tabularnewline
 &  & SALDO & 73'725\tabularnewline
\end{tabular}
\end{verse}

\section*{AHV / IV - Lohnabrechnungen}
\begin{verse}
\begin{tabular}{|c|c|c|r|}
\hline 
Text & Soll & Haben & Betrag\tabularnewline
\hline 
\hline 
Banküberweisung der Nettolöhne & Lohnaufwand & Bank  & 220'000\tabularnewline
\hline 
Arbeitnehmerbeiträge & Lohnaufwand & Kred. Sozialversichers. & 44'500\tabularnewline
\hline 
Arbeitgeberbeiträge & Sozialaufwand & Kred. Sozialversichers. & 45'100\tabularnewline
\hline 
\end{tabular}
\end{verse}

\section*{Lohnabrechnung Details}
\begin{verse}
\begin{tabular}{|l|r|r|r|}
\hline 
\textbf{Lohnabrechnungen} & \textbf{Teilhaber A} & \textbf{Teilhaber B} & \textbf{Teilhaber C}\tabularnewline
\hline 
\hline 
Bruttolohn (BL) & 60'000.00 & 60'000.00 & 60'000.00\tabularnewline
- Koordinationsabzug & -24'360.00 & -24'360.00 & -24'360.00\tabularnewline
= Versicherter Lohn (VL) & 35'640.00 & 35'640.00 & 35'640.00\tabularnewline
- AHV/IV/EO/ALV 6.25\% vom BL & -3'750.00 & -3'750.00 & -3'750.00\tabularnewline
- Pensionskasse 7.00\% vom VL & -2'494.80 & -2'494.80 & -2'494.80\tabularnewline
\hline 
= Nettolohn & 53'755.20 & 53'755.20 & 53'755.20\tabularnewline
\hline 
\end{tabular}

\begin{tabular}{|l|r|r|r|r|}
\hline 
\textbf{Arbeitnehmerbeiträge} & \textbf{Total} & \textbf{Teilhaber A} & \textbf{Teilhaber B} & \textbf{Teilhaber C}\tabularnewline
\hline 
\hline 
AHV/IV/EO/ALV & 11'250.00 & 3'750.00 & 3'750.00 & 3'750.00\tabularnewline
\hline 
Pensionskasse (1) & 7'484.40 & 2'494.80 & 2'494.80 & 2'494.80\tabularnewline
\hline 
\end{tabular}

\begin{tabular}{|l|r|r|r|r|}
\hline 
\textbf{Arbeitgeberbeiträge} & \textbf{Total} & \textbf{Teilhaber A} & \textbf{Teilhaber B} & \textbf{Teilhaber C}\tabularnewline
\hline 
\hline 
AHV/IV/EO/ALV & 11'250.00 & 3'750.00 & 3'750.00 & 3'750.00\tabularnewline
\hline 
Pensionskasse (2) & 8'553.60 & 2'851.20 & 2'851.20 & 2'851.20\tabularnewline
\hline 
\end{tabular}\end{verse}
\begin{itemize}
\item (1) 7.00\% vom versichterten Lohn
\item (2) 8.00\% vom versicherten Lohn
\item Beitragssätze (7.00\% bzw. 8.00\% differieren je nach Versicherung
und Arbeitgeber)
\end{itemize}

\section*{MWST inkl. Skonto - Rechnung}
\begin{verse}
\begin{tabular}{r|>{\raggedright}p{4cm}|r|l|c|r|r|}
\cline{2-7} 
 & Text & Soll & Haben & Betrag & \multicolumn{2}{c|}{Prozent}\tabularnewline
\hline 
\multirow{2}{*}{100 \% = 432'000} & An Kunden verrechnet Beratunsertrag & Debitoren & Beratungsertrag & 400'000 & 100 \% & \multirow{2}{*}{108 \%}\tabularnewline
\cline{2-6} 
 & MWST fakturieren & Debitoren & Kred. Umsatzsteuer & 32'000 & 8 \% & \tabularnewline
\hline 
98 \% = 423'360  & Kundenzahlung Postkonto inkl MWST & Postkonto & Debitoren & 423'360 & \multicolumn{2}{r|}{}\tabularnewline
\hline 
\multirow{2}{*}{2 \% = 8'640 } & Skontoabzug des Kunden  & Debitorenverluste & Debitoren & 8000 & 100 \%  & \multirow{2}{*}{108 \%}\tabularnewline
\cline{2-6} 
 & inkl. MWST & Kred. Umsatzsteuer & Debitoren & 640 & 8 \% & \tabularnewline
\hline 
\end{tabular}
\end{verse}

\paragraph*{Umsatzsteuer (keine Gewinnsteuer): }

Konto Kreditor (dem Staat)


\section*{MWST inkl. Skonto - Einkauf}
\begin{verse}
\begin{tabular}{r|>{\raggedright}p{4cm}|r|l|c|r|r|}
\cline{2-7} 
 & Text & Soll & Haben & Betrag & \multicolumn{2}{c|}{Prozent}\tabularnewline
\hline 
\multirow{2}{*}{100 \% = 432'000} & Rechnung vom Kunden & Warenaufwand & Kreditor & 400'000 & 100 \% & \multirow{2}{*}{108 \%}\tabularnewline
\cline{2-6} 
 & MWST fakturieren & Debitor Vorsteuer & Kreditor & 32'000 & 8 \% & \tabularnewline
\hline 
98 \% = 423'360  & Zahlung per Postkonto & Kreditor & Post & 423'360 & \multicolumn{2}{r|}{}\tabularnewline
\hline 
\multirow{2}{*}{2 \% = 8'640 } & Skontoabzug von uns & Kreditor & Warenaufwand & 8000 & 100 \%  & \multirow{2}{*}{108 \%}\tabularnewline
\cline{2-6} 
 & inkl. MWST & Kreditor & Debitor Vorsteuer & 640 & 8 \% & \tabularnewline
\hline 
\end{tabular}
\end{verse}

\paragraph*{Vorsteuer: }

Konto Debitor (vom Staat)


\section*{Reserven aus Gewinn bilden}
\begin{verse}
\begin{tabular}{|l|r|c|l|c|}
\hline 
Text & Soll &  & Haben & Betrag\tabularnewline
\hline 
\hline 
Vom Jahresgewinn werden CHF 10'000.- den Reserven zugewiesen & Bilanzgewinn & / & Reserven & 10'000\tabularnewline
\hline 
\end{tabular}
\end{verse}

\section*{Begriffe}


\paragraph*{Anlagen:}
\begin{itemize}
\item langfristig materiell: Immobilien, Mobilien etc.
\item langfristig immateriell: Softwarelizenzen, Patente
\end{itemize}

\paragraph*{Diverse}
\begin{itemize}
\item übrige kurzfristige Verbindlichkeiten: auf Kredit
\item Kundenrechnungen, Kundenforderungen: Debitoren
\item auf Rechnung: Kreditoren, Verpflichtungen aus Lieferungen und Leistungen,
übriges kurzfristiges Fremdkapital\end{itemize}

