
\part*{Bewertung}


\section*{Grundlagen}

In den meissten Fällen ist es nicht möglich den künftigen Nutzenzugang
zu quantifizieren. Zudem ist auch nicht bestimmt, mit welchem kalkulatorischen
Zinssatz die Nutzenzugänge zu diskontieren sind. Der Gesetzgeber legt
Höchstwerte fest, über die hinaus Aktiven nicht bewertet werden dürfen
(Gläubigerschutz). Siehe Gesetzesartikel.

Gesetzliche Bewertungsgrundsätze:
\begin{quote}
\begin{tabular}{|l|l|l|}
\hline 
\noun{Vorsicht} & \noun{Stetigkeit} & \noun{Fortführung}\tabularnewline
\hline 
\hline 
Es soll vorsichtig, eher in schlechtem  & Bewertungsgrundsätze über  & Bewertet aus der Perspektive der \tabularnewline
Licht bewertet werden. & mehrere Abschlüsse beibehalten. & Weiterexistenz des Unternehmens.\tabularnewline
\hline 
\end{tabular}
\end{quote}
Weiterführung des Vorsichtsprinzips:
\begin{quote}
\begin{tabular}{|c||>{\raggedright}p{10cm}|}
\hline 
\noun{Niederstwertprinzip} & Sind vom Gesetz verschiedene Wertansätze zugelassen, so ist der tiefste
von allen zu wählen.\tabularnewline
\hline 
\noun{Realisationsprinzip} & Gewinne sollen erst verbucht werden, wenn sie realisiert werden\tabularnewline
\hline 
\noun{Imparitätsprinzip} & Nicht realisierte Gewinne dürfen nicht verbucht werden, mutmassliche
Verluste müssen hingegen erfasst werden.\tabularnewline
\hline 
\end{tabular}
\end{quote}

\section*{Bewertung Anlagevermögen}
\begin{itemize}
\item Im Allgemeinen

\begin{itemize}
\item Das Anlagevermögen darf höchstens zu den Anschaffungs- oder den Herstellungskosten
bewertet werden, unter Abzug der notwendigen Abschreibungen.
\end{itemize}
\item Beteiligungen

\begin{itemize}
\item Stimmberechtigte Anteile von mindestens 20 Prozent gelten als Beteiligung.
(Bewertung gemäss allgemeinem Anlagevermögen)
\end{itemize}
\item Vorräte (Rohmaterial, Teil- und Fertigfabrikate)

\begin{itemize}
\item Dürfen höchstens zu Anschaffungs- oder Herstellungskosten bewertet
werden.
\item Sind die Kosten am Bilanzstichtag höher als der aktuelle Marktpreis,
so ist dieser massgebend.
\end{itemize}
\item Wertschriften

\begin{itemize}
\item Wertschriften mit Kurswert dürfen höchstens zum Durchschnittskurs
des letzten Monats vor dem Bilanzstichtag bewertet werden.
\item Wertschriften ohne Kurswert dürfen höchstens zu den Anschaffungskosten
bewertet werden, unter Abzug der notwendigen Wertberichtigung.
\end{itemize}
\end{itemize}

\section*{Anschaffungswert}

\begin{tabular}{l}
Anschaffung\tabularnewline
+ Bezugs- bzw. Transportkosten\tabularnewline
+ Installationskosten\tabularnewline
+ Eventuelle Kosten des Betriebsunterbruchs\tabularnewline
- Allfällige Preisnachlässe (Rabatt/Skonto)\tabularnewline
\hline 
= Anschaffungswert\tabularnewline
\end{tabular}


\section*{Rechnungslegungsnormen}

Ziel: höchst möglicher Gläubigerschutz zu erzielen

Dadurch wird aber die Transparenz für den Investor vermindert. Für
den Investor taugen nur die Abschlussrechnungen, die nach dem ``true
and fair-view'' Prinzip erstellt worden sind und wahre und objektiv
richtige Wertansätze enthalten. Verschiedene Rechnungslegungsnormen:
SWISS\_GAAP\_FER / IFRS / US\_GAAP
