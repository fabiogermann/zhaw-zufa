
\section*{Kombinatorische Schaltungen vs Sequenzielle Schaltungen}
\begin{itemize}
\item Komb: Reine Schaltnetze, werden nur durch Input beeinflusst, Zeiteinfluss
nur von int. Delays
\item Seq: Haben ``Speicher'', Zustände werden auch von den vorhergehenden
Inputs beeinflusst (Zustand des Speichers), Zeitlicher Ablauf spielt
eine grosse Rolle
\end{itemize}

\subsection*{D-Latch (transp. FF)}
\begin{itemize}
\item D = Input
\item C = Control (solange C=1, wird der Input durchgeschaltet)
\item Q !Q = Outputs
\item Rückkopplung ist gefährlich, gespeicherter Zustang zufällig
\end{itemize}

\subsection*{D-Flip-Flop (nichttransp. FF)}
\begin{itemize}
\item Pos/Neg Flanken gesteuert
\item gleiche Funktionsweise wie D-Latch
\item Rückkopplung halbiert den Takt
\item Set 1 und Reset 0 Funktion -> Synchron mit Takt oder Asynchron sofort
\end{itemize}

\subsection*{Bausteine}
\begin{itemize}
\item Register: aus FF mit write-enable: zur Datenspeicherung
\item asynchrone Schaltungen: D-FF ohne Clock, Staffelung, arbeiten so schnell
wie möglich\end{itemize}

