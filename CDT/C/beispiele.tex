
\section*{Beispiel Code}

:-)

\begin{tabular}{|l|l|}
\hline 
\#include <stdio.h> & \tabularnewline
\hline 
\#include <string.h> & /{*}Enthält: int strlen(const char s{[}{]}); welche die länge des
Strings zurückgibt(ohne \textbackslash{}0){*}/\tabularnewline
\hline 
\hline 
\textcolor{red}{int max(int a, int b);} & \tabularnewline
\hline 
\textcolor{red}{void otherFunction(void);} & \textcolor{red}{/{*}void soll geschrieben werden, damit Parameter
checking durchgeführt wird{*}/}\tabularnewline
\hline 
\textcolor{red}{int checkStatic(int a);} & \tabularnewline
\hline 
\hline 
\#define MAX\_LENGTH 1000  & /{*}Definieren einer Konstante{*}/\tabularnewline
\hline 
\hline 
\textcolor{blue}{typedef enum \{Mo, Di...\} Wochentage} & \textcolor{blue}{/{*}Montag = 0, Dienstag = 1 etc{*}/}\tabularnewline
\hline 
\textcolor{blue}{typdef enum {[}false, true{]}\} Bool;} & \tabularnewline
\hline 
\hline 
\textcolor{green}{typedef struct \{} & \tabularnewline
\hline 
\textcolor{green}{int x;} & \tabularnewline
\hline 
\textcolor{green}{int y;} & \tabularnewline
\hline 
\textcolor{green}{int z;} & \tabularnewline
\hline 
\textcolor{green}{\} Point3D} & \tabularnewline
\hline 
\hline 
\textcolor{magenta}{int min = 0; } & \textcolor{magenta}{/{*}Definition globale Variable {*}/}\tabularnewline
\hline 
\hline 
int main(void) \{ & \tabularnewline
\hline 
int i = 5, a=4, b=6, len; & \tabularnewline
\hline 
double d; & \tabularnewline
\hline 
\hline 
\textcolor{magenta}{extern int min; } & \textcolor{magenta}{/{*}Deklaration globale Variable {*}/}\tabularnewline
\hline 
\hline 
\textcolor{blue}{Bool flag = true;} & \tabularnewline
\hline 
\hline 
int data{[}100{]}; & /{*}Array Deklaration{*}/\tabularnewline
\hline 
int a{[}5{]} = \{4,7,12,77,2\}; & /{*}Array Deklaration + Initialisierung{*}/\tabularnewline
\hline 
int a{[}{]} = \{4,7,12,77,2);  & /{*}Alternative Array Deklaration + Initialisierung {*}/\tabularnewline
\hline 
const int c{[}{]} = \{10,11,12,13,14\}; & /{*}Konstante Array Deklaration, Werte können NICHT mehr verändert
werden{*}/\tabularnewline
\hline 
int a{[}2{]}{[}3{]} = \{\{1,2,3\},\{4,5,6\}\}; & /{*}Array Deklaration mit 2 Dimensionen (2 Zeilen, 3 Spalten){*}/\tabularnewline
\hline 
char hello{[}{]} = „hello“;  & /{*}Deklaration eines Strings (Array von Chars)(Grösse +1, da letzes
Element: \textbackslash{}0){*}/\tabularnewline
\hline 
char test{[}{]} = „test“; & \tabularnewline
\hline 
a{[}3{]} = 4; & /{*}Zugriff auf Array Element{*}/\tabularnewline
\hline 
a{[}9{]} = 222;  & /{*}Achtung! Es wird kein Fehler zur Laufzeit angezeigt!!{*}/\tabularnewline
\hline 
len = strlen(hello);  & /{*}len = 5 {*}/\tabularnewline
\hline 
strcat(hello, test); & /{*}test wird an hello angefügt: char hello{[}{]} = „hellotest“{*}/\tabularnewline
\hline 
strcpy(char dest{[}{]}, const char source{[}{]}); & /{*}Kopieren eines Strings (char{*}) {*}/\tabularnewline
\hline 
strcmp(const char s1{[}{]}, const char s2{[}{]}); & /{*}Vergleicht zwei Strings, gibt zurück: <0, wenn s1 kl., >0 wenn
s2 kl., 0 wenn gl.{*}/\tabularnewline
\hline 
\hline 
int j; & \tabularnewline
\hline 
int {*}jp; & \tabularnewline
\hline 
jp = \&j; & /{*} Zuweisung der Adresse von j; jp zeigt jetzt auf j{*}/\tabularnewline
\hline 
{*}jp = 3; & /{*} jp wird dereferenziert und dem Objekt wird 3 zugewiesen; j ist
jetzt also 3{*}/\tabularnewline
\hline 
void {*}vp; & /{*} Pointer von Typ void, dieser kann auf irgendetwas zeigen{*}/\tabularnewline
\hline 
jp = NULL;  & /{*} Pointer zeigt explizit „auf nichts“ {*}/\tabularnewline
\hline 
int h{[}3{]} = \{2,4,6\}; & \tabularnewline
\hline 
int {*}pa; & \tabularnewline
\hline 
pa = h; & /{*} Pointer zeigt nun auf den Array a, genauer auf das erste Element
a{[}0{]}{*}/\tabularnewline
\hline 
{*}pa = 7; & /{*} a{[}0{]} ist nun 7{*}/\tabularnewline
\hline 
{*}(pa + 3) = 8; & /{*}a{[}3{]} ist nun 8 \{pa+i → pa zeigt auf i-te Element) {*}(pa+3)
ist äquiv. zu pa{[}3{]}{*}/\tabularnewline
\hline 
{*}pa = 8; & /{*} a{[}0{]} ist nun 8{*}/\tabularnewline
\hline 
pa++; & /{*} Pointer zeigt nun auf das nächste Element im Array{*}/\tabularnewline
\hline 
char a{[}{]} = „hello, Winterthur“; & \tabularnewline
\hline 
char {*}pa = „hello, Switzerland“; & \tabularnewline
\hline 
{[}a = pa;{]} & /{*} Nicht möglich → Kompilierfehler{*}/\tabularnewline
\hline 
pa = a; & /{*}OK, Pointer zeigt nun auf „hello, Winterthur“ {*}/\tabularnewline
\hline 
char{*} pmonth{[}12{]} = \{„Jan“, „Feb“, …\} & /{*}Pointer auf Array, Anstatt 2-Dimensionaler Array{*}/\tabularnewline
\hline 
pmonth{[}1{]}; & /{*}Greift auf February zu{*}/\tabularnewline
\hline 
{*}(pmonth{[}1{]}+3); & /{*}Greift auf das ‚r‘ in February zu{*}/\tabularnewline
\hline 
pmonth{[}1{]}{[}3{]}; & /{*}Greift ebenfalls auf das ‚r‘ in February zu{*}/\tabularnewline
\hline 
\hline 
\textcolor{green}{Point3D pt = \{2, 4, 6\};} & \tabularnewline
\hline 
\textcolor{green}{(void)printf(„A=(\%d, \%d, \%d)\textbackslash{}n“} & \tabularnewline
\textcolor{green}{,pt.x,pt.y,pt.z);} & \tabularnewline
\hline 
\end{tabular}

\begin{tabular}{|l|l|}
\hline 
d = i/3;  & /{*} d= 1.0 {*}/\tabularnewline
\hline 
d = (double) i/3;  & /{*}d = 1.66667{*}/\tabularnewline
\hline 
\hline 
\textcolor{red}{i = max(a,b);} & \textcolor{red}{/{*}i = b = 6{*}/}\tabularnewline
\hline 
ohterFunction(); & \tabularnewline
\hline 
\textcolor{blue}{Wochtage w1 = Mittwoch;} & \tabularnewline
\hline 
(void)printf(„Hello World in C\textbackslash{}n“); & \tabularnewline
\hline 
i=scanf(„\%d\%d\%d“,\&day,\&month,\&year); & \tabularnewline
\hline 
(void)printf(“\%d”, day); & \tabularnewline
\hline 
\hline 
for (i = 1; i<=max; i++) \{  & /{*}Deklaration int i = 1 geht in c nicht, i muss vorher schon deklariert
werden {*}/\tabularnewline
\hline 
/{*}do something{*}/ & \tabularnewline
\hline 
\} & \tabularnewline
\hline 
\hline 
switch(n) \{ & \tabularnewline
\hline 
case 1: result = 1; & \tabularnewline
\hline 
break & \tabularnewline
\hline 
case 2: result = 2; & \tabularnewline
\hline 
break & \tabularnewline
\hline 
default: result =3; & \tabularnewline
\hline 
break & \tabularnewline
\hline 
\} & \tabularnewline
\hline 
\hline 
exit(0); & \tabularnewline
\hline 
\} & \tabularnewline
\hline 
\hline 
\textcolor{red}{int max(int a, int b) \{} & \tabularnewline
\hline 
if(a<b) \{ & \tabularnewline
\hline 
return b; & \tabularnewline
\hline 
\} & \tabularnewline
\hline 
return a; & \tabularnewline
\hline 
\textcolor{red}{\}} & \tabularnewline
\hline 
\hline 
\textcolor{red}{void otherFunction(void) \{} & \tabularnewline
\hline 
\textcolor{magenta}{extern int max; } & \textcolor{magenta}{/{*}Deklaration glob. Variable {*}/}\tabularnewline
\hline 
int i = 8;  & /{*}Entsteht kein Konflikt mit i aus main {*}/\tabularnewline
\hline 
otherFunction(void); & /{*} Rekursion, wie in Java{*}/\tabularnewline
\hline 
\textcolor{red}{\}} & \tabularnewline
\hline 
\hline 
\textcolor{red}{int checkStatic(int a) \{} & \tabularnewline
\hline 
static int max = 0;  & /{*}Deklaration statischer Variable{*}/\tabularnewline
\hline 
if (a > max) \{ & /{*} Statische Variable {*}/\tabularnewline
\hline 
max = a; & /{*} lokale Variable {*}/\tabularnewline
\hline 
\} & \tabularnewline
\hline 
return max; & \tabularnewline
\hline 
\textcolor{red}{\}} & \tabularnewline
\hline 
\end{tabular}
