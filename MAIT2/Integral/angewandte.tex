
\section*{Angewandte Integrale}


\subsection*{Fläche zwischen Funktionen}

\begin{tabular}{|c|c|c|}
\hline 
$f$ oberhalb $g$ & $g$und $f$ schneiden sich, $x_{i}$ Schnittpunkte & Mantelfläche\tabularnewline
\hline 
$A=\intop_{a}^{b}f(x)dx-\int_{a}^{b}g(x)dx$ & $A=|\int_{a}^{x_{1}}(f-g)(x)dx|+|\intop_{x_{1}}^{x_{2}}(f-g)(x)dx|+...$ & $M_{x}=2\pi\times\intop_{a}^{b}f(x)\times\sqrt{1+(f'(x))^{2}}dx$\tabularnewline
\hline 
\multicolumn{1}{c}{} & \multicolumn{1}{c}{} & \multicolumn{1}{c}{}\tabularnewline
\hline 
Rotation um die X-Achse (Volumen) & Volumen bei Querfläche & Bogenlänge\tabularnewline
\hline 
$V_{x}=\pi\intop_{a}^{b}f(x)^{2}dx$ & $V=\int_{a}^{b}Q(x)dx$ & $S=\int_{a}^{b}\sqrt{1+(f'(x))^{2}}dx$\tabularnewline
\hline 
\multicolumn{1}{c}{} & \multicolumn{1}{c}{} & \multicolumn{1}{c}{}\tabularnewline
\hline 
Schwerpunkt einer Fläche & Schw. e. Fl. zwischen $g(x)$ und $f(x)$, $g(x)\leq f(x)$ in I & Schwerpunkt eines Rotationskörpers\tabularnewline
\hline 
$S_{x}=\frac{\int_{a}^{b}x\times f(x)dx}{A}$ & $S_{x}=\frac{\int_{a}^{b}x\times(f(x)-g(x))dx}{F}$ & $S_{x}=\frac{\pi\int_{a}^{b}x\times f^{2}(x)dx}{V}$\tabularnewline
\hline 
$S_{y}=\frac{\frac{1}{2}\int_{a}^{b}f(x)^{2dx}}{A}$ & $S_{y}=\frac{\frac{1}{2}\int_{a}^{b}(f^{2}(x)-g^{2}(x))dx}{F}$ & $S_{y}=0$, $S_{z}=0$\tabularnewline
\hline 
$A=\int_{a}^{b}f(x)dx$ & $F=\int_{a}^{b}(f(x)-g(x))dx$ & $V=\pi\int_{a}^{b}f^{2}(x)dx$\tabularnewline
\hline 
\end{tabular}


