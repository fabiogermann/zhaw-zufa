
\part*{Aussagelogik und Elementare Mengenlehre}


\section*{Aussagelogik}


\subsection*{Aussageverbeindungen}
\begin{lyxlist}{00.00.0000}
\item [{%
\begin{tabular}{ll|cl|ll}
$\wedge$ & AND (Konjunktion) & $\Rightarrow$ & Implikation & $\exists$ & Es gibt\tabularnewline
$\vee$ & OR (Disjunktion) & $\Leftrightarrow$ & Äquivalenz & $\exists!$ & Es gibt genau ein\tabularnewline
$\neg$ & NOT &  &  & $\forall$ & Für alle gilt\tabularnewline
\end{tabular}}]~
\end{lyxlist}

\subsection*{Wahrheitstabellen}

\begin{tabular}{|c|c|c|c|c|c|}
\hline 
$A$ & $B$ & $A\wedge B$ & $A\vee B$ & $A\implies B$ & $A\leftrightarrow B$\tabularnewline
\hline 
\hline 
0 & 0 & 0 & 0 & 1 & 1\tabularnewline
\hline 
0 & 1 & 0 & 1 & 1 & 0\tabularnewline
\hline 
1 & 0 & 0 & 1 & 0 & 0\tabularnewline
\hline 
1 & 1 & 1 & 1 & 1 & 1\tabularnewline
\hline 
\end{tabular}


\subsection*{Umformungen}

Logische Operationen:
\begin{verse}
\begin{tabular}{|l|l|l|l}
\cline{1-3} 
$A\Rightarrow B$ & $\Leftrightarrow$ & $(\neg A)\vee B$ & Wenn A richtig ist, muss B auch richtig sein\tabularnewline
\cline{1-3} 
$\neg(A\Rightarrow B)$ & $\Leftrightarrow$ & $A\wedge(\neg B)$ & Wenn A richtig ist, darf B nicht richtig sein\tabularnewline
\cline{1-3} 
$(A\Rightarrow B)$ & $\Leftrightarrow$ & $(\neg A\Rightarrow\neg B)$ & \tabularnewline
\cline{1-3} 
$(A\Rightarrow B)\wedge A$ & $\Leftrightarrow$ & $B$ & \tabularnewline
\cline{1-3} 
$(A\Rightarrow B)\wedge\neg B$ & $\Leftrightarrow$ & $\neg A$ & \tabularnewline
\cline{1-3} 
$(A\Rightarrow B)\wedge(B\Rightarrow C)$ & $\Leftrightarrow$ & $(A\Rightarrow C)$ & \tabularnewline
\cline{1-3} 
$\neg(A\wedge B)$ & $\Leftrightarrow$ & $(\neg A)\vee(\neg B)$ & A oder B muss falsch sein, damit das ganze richtig ist\tabularnewline
\cline{1-3} 
$\neg(A\vee B)$ & $\Leftrightarrow$ & $(\neg A)\wedge(\neg B)$ & A und B müssen falsch sein, damit das ganze richtig ist\tabularnewline
\cline{1-3} 
$(A\vee B)\wedge\neg A$ & $\Leftrightarrow$ & $B$ & \tabularnewline
\cline{1-3} 
\end{tabular}
\end{verse}
Zusätzlich:
\begin{verse}
\begin{tabular}{|c|c|c|c|c|c|c|c|}
\hline 
OR & $(\neg A\implies B)$ & $\Leftrightarrow$ & $(\neg B\implies A)$ & $\Leftrightarrow$ & $(A\vee B)$ & $\Leftrightarrow$ & \tabularnewline
\hline 
XOR & $(A\wedge\neg B)\vee(\neg A\wedge B)$ & $\Leftrightarrow$ & $(A\vee B)\wedge(\neg A\vee\neg B)$ & $\Leftrightarrow$ & $(A\dot{\vee}B)$ & $\Leftrightarrow$ & $\neg(A\leftrightarrow B)$\tabularnewline
\hline 
XNOR & $(A\wedge B)\vee(\neg A\wedge\neg B)$ & $\Leftrightarrow$ & $(A\vee\neg B)\wedge(\neg A\vee B)$ & $\Leftrightarrow$ & $(A\Leftrightarrow B)$ & $\Leftrightarrow$ & $(A\Rightarrow B)\wedge(B\Rightarrow A)$\tabularnewline
\hline 
\end{tabular}
\end{verse}

\section*{Elementare Mengenlehre}


\subsection*{Mengenoperationen}
\begin{lyxlist}{00.00.0000}
\item [{$\cap$}] Durschschnitt / Schnittmenge
\item [{$\cup$}] Vereinigung
\item [{$\setminus$}] Restmenge
\end{lyxlist}
\begin{tabular}{ccc}
$\cap$ & $B$ & $\bar{B}$\tabularnewline
$A$ & \includegraphics[scale=0.5]{\string"Aussagelogik und Elementare Mengenlehre/Durchschnitt1\string".png} & \includegraphics[scale=0.5]{\string"Aussagelogik und Elementare Mengenlehre/Durchschnitt2\string".png}\tabularnewline
$\bar{A}$ & \includegraphics[scale=0.5]{\string"Aussagelogik und Elementare Mengenlehre/Durchschnitt3\string".png} & \includegraphics[scale=0.5]{\string"Aussagelogik und Elementare Mengenlehre/Durchschnitt4\string".png}\tabularnewline
\end{tabular}%
\begin{tabular}{ccc}
$\cup$ & $B$ & $\bar{B}$\tabularnewline
$A$ & \includegraphics[scale=0.5]{\string"Aussagelogik und Elementare Mengenlehre/Vereinigung1\string".png} & \includegraphics[scale=0.5]{\string"Aussagelogik und Elementare Mengenlehre/Vereinigung2\string".png}\tabularnewline
$\bar{A}$ & \includegraphics[scale=0.5]{\string"Aussagelogik und Elementare Mengenlehre/Vereinigung3\string".png} & \includegraphics[scale=0.5]{\string"Aussagelogik und Elementare Mengenlehre/Vereinigung4\string".png}\tabularnewline
\end{tabular}


\subsection*{Gesetze}

\begin{tabular}{lccc|ccc}
Kommutativgesetz & $A\cup B$ & $=$ & $B\cup A$ & $a+b$ & $=$ & $b+a$\tabularnewline
Assoziativgesetz & $A\cup(B\cup C)$ & $=$ & $(A\cup B)\cup C$ & $a+(b+c)$ & $=$ & $(a+b)+c$\tabularnewline
Dristributivgesetzt & $A\cup(B\cap C)$ & $=$ & $(A\cup B)\cap(A\cup C)$ & $a*(b+c)$ & $=$ & $ab+ac$\tabularnewline
Neutralelement & $\exists x\forall a$: & $=$ & \multicolumn{1}{c}{$x+a=a+x=a$} &  &  & \tabularnewline
Inverses & $\forall x\exists y$: & $=$ & \multicolumn{1}{c}{$x+y=0$} &  &  & \tabularnewline
Gruppe & $\exists Neutralelement$ & $\wedge$ & \multicolumn{1}{c}{$\exists Inverses$} &  &  & \tabularnewline
\end{tabular}


\subsection*{Produktmenge}

$A=\{a,b,c\}$ und $B=\{1,2\}$, dann ist die Produkmenge $AxB=\{a1,a2,b1,b2,c1,c2\}$.


\subsection*{Potenzmenge}

$A=\{a,b\}$, dann ist die Potenzmenge $P(A)=\{\{\},\{a\},\{b\},\{a,b\}\}$


\subsection*{Bespiel Mengenlehre}
\begin{quote}
Gegeben seien die Mengen A (30 Elemente), B (50 Elemente) und C (60
Elemente).

Wie viele Elemente enhält $B\setminus(A\cup C)$, falls $A\cap B$,
$A\cap C$, $B\cap C$ je 5 Elemente und $A\cup B\cup C$ 127 Elemente
enhalten?
\end{quote}
\begin{center}
\includegraphics[width=5cm]{\string"Aussagelogik und Elementare Mengenlehre/Mengenlehre_Beispiel\string".png}
\par\end{center}
\begin{quote}
Gesucht, die Menge X$(A\cap B\cap C)$, mit Hilfe derer man alle Teilmengen
bestimmen kann.

X kann nun wie folgt bestummen werden:

$A\cup B\cup C=A+B+C-A\cap B-A\cap C-B\cap C+X$

Einsetzen:$\rightarrow127=30+50+60-5-5-5+X$

$X=2$ 

$\Rightarrow(B\cap C)\setminus X=3$

$\Rightarrow(B\cap C)\setminus X=3$

$\Rightarrow B\setminus(A\cup C)=42$\end{quote}

