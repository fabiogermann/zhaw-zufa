
\part*{Folgen und Reihen}


\section*{Folgen}

Eine reelle Zahlenfolge $a_{1},a_{2},a_{3},...$, die nach irgendeiner
Vorschrift geordnet sind. Die Folge kann endlich viele Glieder haben
(abbrechende Folge) oder unendlich viele Glieder umfassen.

Bsp: $1,4,9,16...$


\subsection*{Summenzeichen}

$\overset{n}{\underset{k=m}{\sum}}a_{k}=a_{m}+a_{m+1}+a_{m+2}+...+a_{n}$
= ``Summe aller $a_{k}$ von $k=m$ bis $k=n$''

Zu jeder Folge $a_{1},a_{2},a_{3},...$ kann man die Folge $s_{n}$der
Teilsummen, die sogenannte \textbf{Reihe} der Folge bilden:
\begin{itemize}
\item $s_{1}=a_{1}$ 
\item $s_{2}=a_{1}+a_{2}$ 
\item $s_{3}=a_{1}+a_{2}+a_{3}$ 
\item $s_{n}=a_{1}+a_{2}+a_{3}+...+a_{n}=\overset{n}{\underset{k=1}{\sum}}a_{k}$ 
\end{itemize}

\subsection*{Produktzeichen}

$\overset{n}{\underset{k=m}{\prod}}a_{k}=a_{m}\cdot a_{m+1}\cdot a_{m+2}\cdot...\cdot a_{n}$
= ``Produkt aller $a_{k}$ von $k=m$ bis $k=n$''


\subsection*{Rechenregeln}
\begin{verse}
\begin{tabular}{|l|l|l|}
\hline 
c sei konstant: & $\overset{n}{\underset{k=m}{\sum}}c=(n-m+1)\cdot c$ & $\overset{n}{\underset{k=m}{\prod}}c=c^{n-m+1}$\tabularnewline
\hline 
c = Konstanter Faktor: & $\overset{n}{\underset{k=m}{\sum}}c\cdot a_{k}=\overset{n}{c\cdot\underset{k=m}{\sum}}a_{k}$ & $\overset{n}{\underset{k=m}{\prod}}c\cdot a_{k}=c^{n-m+1}\cdot\overset{n}{\underset{k=m}{\prod}}a_{k}$\tabularnewline
\hline 
Zerlegung: & $\overset{n}{\underset{k=m}{\sum}}(a_{k}\pm b_{k})=\overset{n}{\underset{k=m}{\sum}}a_{k}\pm\overset{n}{\underset{k=m}{\sum}}b_{k}$ & $\overset{n}{\underset{k=m}{\prod}}(a_{k}\cdot b_{k})=\overset{n}{\underset{k=m}{\prod}}a_{k}\cdot\overset{n}{\underset{k=m}{\prod}}b_{k}$\tabularnewline
\hline 
\end{tabular}
\end{verse}
Beispiele:
\begin{itemize}
\item $\overset{25}{\underset{k=5}{\sum}}a_{k}=\overset{25}{\underset{k=1}{\sum}}a_{k}-\overset{4}{\underset{k=1}{\sum}}a_{k}$
\item $\overset{5}{\underset{k=3}{\sum}}(i^{2}-3)=\overset{5}{\underset{k=3}{\sum}}i^{2}+\overset{5}{\underset{k=3}{\sum}}-3$
\end{itemize}

\section*{Arithmetische Folgen}

Eine Folge bei der die Differenz d zweier aufeinander folgender Glieder
konstant ist heisst arithmetische Folge (AF).
\begin{verse}
$a_{1},a_{1}+d,a_{1}+2d,a_{1}+3d,a_{1}+4d,...$
\end{verse}

\subsection*{Rekursive Definition}

Jedes Glied $a_{n}$ ist durch ein oder mehrere Vorgänger definiert:
\begin{verse}
\begin{tabular}{|c|}
\hline 
$a_{n+1}=a_{n}+d$\tabularnewline
\hline 
\end{tabular}
\end{verse}

\subsection*{Explizite Definition}

$a_{n}$ist durch eine Rechnung von n gegeben:
\begin{quote}
\begin{tabular}{|c|c|}
\hline 
$a_{n}=a_{1}+(n-1)\cdot d$ & $d=\frac{a_{i}-a_{k}}{i-k}$\tabularnewline
\hline 
\end{tabular}
\end{quote}

\subsection*{Summen von arithmetischen Folgen}

Die Summe eine Arithmetischen Folge lässt sich wie folgt berechnen:
\begin{verse}
\begin{tabular}{|c|}
\hline 
$s_{n}=\frac{(a_{1}+a_{n})}{2}\cdot n=\frac{1}{2}\cdot n\cdot(a_{1}+a_{n})=n\cdot a_{1}+\frac{n\cdot(n-1)}{2}\cdot d$\tabularnewline
\hline 
\end{tabular}
\end{verse}
Speziell:
\begin{verse}
\begin{tabular}{|c|}
\hline 
$s_{n}=\overset{n}{\underset{k=1}{\sum}}k^{2}=\sum_{k=1}^{n}\sum_{i=k}^{n}i=\frac{n(n+1)(2n+1)}{6}$\tabularnewline
\hline 
\end{tabular}
\end{verse}

\section*{Geometrische Folgen}

Eine Folge, bei der der Quotient zweier aufeinander folgender Glieder
gleich gross ist, heisst geometrische Folge (GF).
\begin{verse}
$a_{1},a_{1}\cdot q,a_{1}\cdot q^{2},a_{1}\cdot q^{3},...$
\end{verse}

\subsection*{Rekursive Definition}

Jedes Glied $a_{n}$ ist durch ein oder mehrere Vorgänger definiert:
\begin{verse}
\begin{tabular}{|c|}
\hline 
$a_{n+1}=a_{n}\cdot q$\tabularnewline
\hline 
\end{tabular}
\end{verse}

\subsection*{Explizite Definition}

$a_{n}$ist durch eine Rechnung von n gegeben:
\begin{quote}
\begin{tabular}{|c|}
\hline 
$a_{n}=a_{1}\cdot q^{n-1}$\tabularnewline
\hline 
\end{tabular}
\end{quote}

\subsection*{Summen von geometrischen Folgen}

Die Summe eine geometrischen Folge lässt sich wie folgt berechnen:
\begin{verse}
\begin{tabular}{|c|}
\hline 
$s_{n}=a_{1}\cdot\frac{1-q^{n}}{1-q}=a_{1}\cdot\frac{q^{n}-1}{q-1}$\tabularnewline
\hline 
\end{tabular}

\begin{tabular}{|c|}
\hline 
$s_{\infty}=a_{1}\cdot\frac{1}{1-q}$\tabularnewline
\hline 
\end{tabular}
\end{verse}

\subsection*{Anwendung in der Finanzmathematik (GF)}


\subsubsection*{Zinseszinsrechnung}

$K_{0}=$Startkapital; $p=$Zinssatz (in \%); $n=$Anzahl Jahre; $K_{n}=$Kapital
nach $n$ Jahren;
\begin{verse}
\begin{tabular}{|c|}
\hline 
$K_{n}=K_{0}\cdot(1+\frac{p}{100})^{n}=K_{0}\cdot q^{n}$\tabularnewline
\hline 
\end{tabular}
\end{verse}
Bemerkung: $q=$Zinsfaktor
\begin{verse}
$q=1+\frac{p}{100}$, also wenn z.B. \noun{$p=6\%\rightarrow q=1.06$ }
\end{verse}

\subsubsection*{Rentenrechnung}

$r=$Rente; $q=$Zinsfaktor; $n=$Anzahl Jahre
\begin{verse}
\begin{tabular}{|c|}
\hline 
$K_{n}=r\cdot\frac{q^{n}-1}{q-1}$\tabularnewline
\hline 
\end{tabular}
\end{verse}
Wenn noch ein Startkapital $K_{0}$vorhanden ist:
\begin{verse}
\begin{tabular}{|c|}
\hline 
$K_{n}=K_{0}\cdot q^{n}+r\cdot\frac{q^{n}-1}{q-1}$\tabularnewline
\hline 
\end{tabular}
\end{verse}

\subsubsection*{Ratenzahlungen}

$K_{0}=$Schuld; $q=$Zinsfaktor; $n=$Anzahl Jahre; $r=$Rate
\begin{verse}
\begin{tabular}{|c|}
\hline 
$K_{0}\cdot q^{n}=r\cdot\frac{q^{n}-1}{q-1}$\tabularnewline
\hline 
\end{tabular}
\end{verse}
Ist nun die höhe der Raten gefragt, so kann der Zinsfaktor und die
Schuld eingesetzt werden, und nach r aufgelöst werden.


\section*{Grenzwerte}


\subsection*{Monotonie}
\begin{itemize}
\item Eine Folge $a_{n}$heisst \textbf{monoton wachsend (streng monoton
wachsend)}, wenn $a_{n}\leq a_{n+1}(a_{n}<a_{n+1})$ ist für alle
$n$
\item Eine Folge $a_{n}$heisst \textbf{monoton fallend (streng monoton
fallend)}, wenn $a_{n}\geq a_{n+1}(a_{n}>a_{n+1})$ ist für alle $n$
\end{itemize}

\subsection*{Beschränktheit}
\begin{itemize}
\item Eine Folge $a_{n}$heisst \textbf{nach oben beschränkt}, wenn es eine
Zahl $S$ gibt, so dass $a_{n}\leq S$ für alle $n$ gilt. $S$ heisst
obere Schranke der Folge. Eine gegen oben beschränkte Folge hat stets
einen Grenzwert. Der Grenzwert ist die \textbf{kleinste obere Schranke}.
\item Eine Folge $a_{n}$heisst\textbf{ nach} \textbf{unten beschränkt},
wenn es eine Zahl $S$ gibt, so dass $a_{n}\geq s$ für alle $n$
gilt. $s$ heisst untere Schranke der Folge.Eine gegen unten beschränkte
Folge hat stets einen Grenzwert. Der Grenzwert ist die \textbf{grösste
untere Schranke.}
\item Hat eine Folge sowohl eine obere als auch eine untere Schranke, so
nennt man sie kurz eine beschränkte Folge.
\end{itemize}

\subsection*{Der Grenzwertbegriff}

Wird eine Folge beliebig fortgesetzt, so nähert sie sich im unendlichen
einem Wert. Dieser Wert wird Grenzwert (Limes) genannt.
\begin{verse}
\begin{tabular}{|c|}
\hline 
$\underset{n\rightarrow\infty}{lim\, a_{n}}=a$\tabularnewline
\hline 
\end{tabular}
\end{verse}
Man sagt, die Folge konvergiert gegen $a$.
\begin{verse}
\begin{tabular}{|c|c|c|l}
\cline{2-3} 
\multicolumn{1}{c|}{\textbf{Exp. Grad}} & Beispiel & Grenzwert & Beschreibung\tabularnewline
\cline{1-3} 
Zähler > Nenner & $\underset{n\rightarrow\infty}{lim\, a_{n}}\frac{n+1}{1}$ & $\infty$ & (Der Zähler geht gegen $\infty$, der Nenner nicht)\tabularnewline
\cline{1-3} 
 & $\underset{n\rightarrow\infty}{lim\, a_{n}}\frac{2n^{2}+1}{n}$ & $\infty$ & (Der Zähler geht schneller gegen $\infty$, als der Nenner)\tabularnewline
\cline{1-3} 
Zähler < Nenner & $\underset{n\rightarrow\infty}{lim\, a_{n}}\frac{2n^{2}+1}{3n^{3}}$ & $0$ & (Der Nenner geht schneller gegen $\infty$, als der Zähler)\tabularnewline
\cline{1-3} 
 & $\underset{n\rightarrow\infty}{lim\, a_{n}}\frac{1}{n}$ & $0$ & (Der Nenner geht gegen $\infty$, der Zähler nicht)\tabularnewline
\cline{1-3} 
Zähler = Nenner & $\underset{n\rightarrow\infty}{lim\, a_{n}}\frac{2n+3}{2n+5}$ & $1$ & Der Grenzwert kann anhand der Faktoren von n abgelesen werden\tabularnewline
\cline{1-3} 
 & $\underset{n\rightarrow\infty}{lim\, a_{n}}\frac{n+1}{2n+1}$ & $\frac{1}{2}$ & \tabularnewline
\cline{1-3} 
 & $\underset{n\rightarrow\infty}{lim\, a_{n}}\frac{2n+1}{n+1}$ & $\frac{2}{1}$ & \tabularnewline
\cline{1-3} 
\end{tabular}
\end{verse}

\subsection*{Bedingung}

Die Zahl $a$ heisst Grenzwert der Folge $a_{n}$, falls gilt:

Zu jedem $\varepsilon>0$ gibt es eine Stelle $n_{\varepsilon}$so,
dass alle $n>n_{\varepsilon}$gilt :
\begin{verse}
\begin{tabular}{|c|}
\hline 
$\mid a_{n}-a\mid<\varepsilon$\tabularnewline
\hline 
\end{tabular}
\end{verse}
Beispiel:

\begin{tabular}{rrlrl}
$a_{n}=\frac{n}{n+1}$,  & $\varepsilon=\frac{1}{100}$ &  & Behauptung: & $\underset{n\rightarrow\infty}{lim\, a_{n}}=1$\tabularnewline
Bedingung: & $\mid a_{n}-1\mid<\frac{1}{100}$ & $\Rightarrow$ & $\mid\frac{n}{n+1}-1\mid<\frac{1}{100}$ & $\mid$gleichnamig machen\tabularnewline
 &  &  & $\mid\frac{n-(n+1)}{n+1}\mid<\frac{1}{100}$ & $\mid$vereinfachen\tabularnewline
 &  &  & $\mid\frac{-1}{n+1}\mid<\frac{1}{100}$ & $\mid$Betrag weglassen\tabularnewline
 &  &  & $\frac{1}{n+1}<\frac{1}{100}$ & $\mid$nach n auflösen\tabularnewline
 &  &  & $99<n$ & Nach 99 Gliedern ist man das erste mal um $\frac{1}{100}$ am Grenzwert
dran.\tabularnewline
\end{tabular}


\subsection*{Konvergenz, Divergenz}
\begin{itemize}
\item Eine Folge heisst \textbf{konvergent}, falls sie \textbf{einen Grenzwert}
hat.
\item Eine Folge heisst \textbf{divergent}, wenn sie \textbf{keinen Grenzwert}
hat.
\end{itemize}

\subsection*{Rechnen mit Grenzen}

$a_{n}$und $b_{n}$seien konvergente Folgen mit $\underset{n\rightarrow\infty}{lim\, a_{n}}=a$
bzw. $\underset{n\rightarrow\infty}{lim\, b}=ab$, dann gilt:
\begin{verse}
\begin{tabular}{|l|l|l|}
\hline 
$\underset{n\rightarrow\infty}{lim\,(a_{n}\pm b_{n})}=a\pm b$ & $a_{n}=\frac{2+\frac{3}{n}}{5+\frac{6}{n}}+\frac{50}{n^{2}}$ & $\underset{n\rightarrow\infty}{lim\, a_{n}}=\frac{2}{5}+0=\frac{2}{5}$\tabularnewline
\hline 
$\underset{n\rightarrow\infty}{lim\,(a_{n}\cdot b_{n})}=a\cdot b$ & $a_{n}=\frac{6n}{2n}\cdot\frac{2n}{n}$ & $\underset{n\rightarrow\infty}{lim\, a_{n}}=3\cdot2=6$\tabularnewline
\hline 
$\underset{n\rightarrow\infty}{lim\,\frac{a_{n}}{b_{n}}}=\frac{a}{b}$
(für $b_{n}$, $b\neq0$) & $a_{n}=\frac{\frac{36n}{n}}{6+\frac{6}{n}}$ & $\underset{n\rightarrow\infty}{lim\, a_{n}}=\frac{36}{6+0}=\frac{36}{6}=6$\tabularnewline
\hline 
$\underset{n\rightarrow\infty}{lim\,(a_{n}^{k})}=a^{k},k\epsilon\mathbb{R}$ & $a_{n}=\frac{6n}{2n}^{3}$ & $\underset{n\rightarrow\infty}{lim\, a_{n}}=3^{3}=27$\tabularnewline
\hline 
\end{tabular}
\end{verse}

\subsection*{Spezielle Grenzwerte}


\subsubsection*{Reihenwerte}

Gegeben sei eine unendliche Folge $a_{n}$. Wir betrachten die zu
dieser Folge gehörige Reihe $s_{n}$. Konvergiert die Folge $s_{n}$,
so definiert man die \textbf{Summe der unendliche Reihe} als:
\begin{verse}
\begin{tabular}{|c|}
\hline 
$s_{\infty}=a_{1}+a_{2}+...=\overset{\infty}{\underset{k=1}{\sum}}a_{k}=\underset{n\rightarrow\infty}{lim\, s_{n}}$\tabularnewline
\hline 
\end{tabular}
\end{verse}
$s_{\infty}$heisst auch der Reihenwert der Folge $a_{n}$.
\begin{itemize}
\item Geometrische Folgen mit $\mid q\mid>1$ sind \textbf{divergent}.
\item Geometrische Folgen mit $\mid q\mid<1$ sind \textbf{konvergent} mit
dem Grenzwert 0.Zudem gilt:\end{itemize}
\begin{verse}
\begin{tabular}{|c|}
\hline 
$s_{\infty}=\overset{\infty}{\underset{k=1}{\sum}}a_{k}=\frac{a_{1}}{1-q}$\tabularnewline
\hline 
\end{tabular}\end{verse}

